
Polarisation is an important property of light, arising from Maxwell equations\cite{weir_optical_2019}. It refers to the direction of oscillation of the electromagnetic field. Even naked human eyes can display some sensitivity to the state of polarisation of light\cite{le_floch_polarization_2010}. However, the perception of polarisation is much more developed within animal kingdom. This led to the development of advanced characterization techniques of animal species such as beetles\cite{carter_variation_2016}. Fine control over polarisation and polarised emitters also have important applications \textit{e.g.} in quantum information technologies\cite{bennett_quantum_2014} and for commercial RealD 3D digital
stereoscopic projection technologies\cite{mendiburu_chapter_2009}.

Chiral structures are a highly studied area\cites{belyakova_optical_2011}{harutyunyan_optical_2007}{mccall_simplified_2009}{mccall_properties_2009} as this kind a photonic structure allows for unusual optical properties. Among them is the ability to produce circularly polarised laser outputs when pumped\cites{topf_modes_2014}{kopp_twist_2002}{oldano_comment_2004}. Chiral structures are often associated whit liquid-crystals media. Those are versatile materials that allow producing laser outputs when pumped \cites{coles_liquid-crystal_2010}{gardiner_paintable_2011}. Such lasers can also be arranged in arrays\cite{hands_two-dimensional_2008}.

This study draws out the work of \textcite{topf_modes_2014}. It extends coupled waves theory to have analytical expressions for left-handed media and every possible interfaces between chiral and isotropic media. Previous results on chiral media are reproduced and three designs of chiral lasers are proposed to produce circularly polarised outputs.

The first chapter of the report introduces the two theories used and the extensions derived during this study. It also offers a short review of past results on chiral media and presents the cavity designs modelled.

The second chapter gives the methods used to characterise the cavities: the polarisation output as well as the light intensity distribution within the media for every identified laser mode.

The third chapter brings the results obtained for each of the cavities, in terms of reflectivity when it makes sense and of laser output. The cavity proposed by \textcite{topf_modes_2014} is reproduced and similar results to theirs are reported. The three other cavities present interesting properties in term of tunability and purity of output.

Finally a conclusion is drawn on this study and more specifically the cavity designs allowing circular polarised outputs. Some future prospects on chiral lasers are outlined.

Several comparisons between the two theories used are available in appendix, as well as a list of the files used to make the figures of this report in appendix \ref{chap:sourcecode}. The files are hosted at \url{https://github.com/Klafyvel/Hybrid-Chiral-Lasers}.