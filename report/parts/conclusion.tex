Exact and coupled-waves theory are extensively presented and coupled-waves theory is extended to include rotations along propagation axis, left-handed media and interfaces. 

Previous results are successfully reproduced and new designs are implemented. First, cavity with a rotational design allows finely tuning the cavity to produce laser output for many cavity detunings. Then, hybrid cavity composed of a chiral gain medium surrounded by two chiral reflectors of opposite handedness allows producing highly pure circularly polarised outputs. Combining both techniques allows taking the best of both worlds and leads to the hybrid chiral defect cavity, a highly tunable circularly polarised laser source. 

The final design presented could very well be at the heart of the design of new kinds of 3D projectors when implemented with liquid-crystals media in laser matrices. This is could also be used to implement broadband characterisation devices based on ellipsometry. 

The coupled-waves theory could also be extended even further, allowing to simulate cavities with physical beams. Indeed, this study focuses on plane-waves that propagate along the axis of the birefringence helix. However, since the effective refractive index depends on the angle of propagation, physical beams such as Gaussian beams would effectively experience different refractive index depending on the position in the $(x,y)$ plane for a given position $z$. This could lead to interesting effects similar to self-focusing in non-linear optics\cites{poy_chirality-enhanced_2020}{shishkov_polarization_2020}.